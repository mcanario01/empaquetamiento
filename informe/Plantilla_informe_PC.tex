%%%%%%%%%%%%%%%%%%%%%%%%%%%%%%%%%%%%%%%%%%%%%%%%%%%%%%%%%%%%%%%%%%%%%%%%%%%
%
% Plantilla para un artículo en LaTeX en español.
%
%%%%%%%%%%%%%%%%%%%%%%%%%%%%%%%%%%%%%%%%%%%%%%%%%%%%%%%%%%%%%%%%%%%%%%%%%%%
\documentclass[12pt,letterpaper]{article}
% Esto es para que el LaTeX sepa que el texto está en español:
\usepackage[spanish]{babel}
% Esto es para poder escribir acentos directamente:
\usepackage[utf8]{inputenc}
\usepackage[right=2cm,left=2cm,top=2cm,bottom=2cm,headsep=10.0pt,footskip=1cm]{geometry}
\usepackage{fancyhdr}
\usepackage{hyperref}
\usepackage{graphicx}			% Para usar figuras JPG, PNG, etc...
\usepackage{multirow} % para unir filas
\usepackage{multicol} % para unir columnas
\usepackage{dcolumn}
\usepackage{float}
\usepackage{subcaption}
\usepackage{listings}
\usepackage{xcolor}

% PORTADA
\lhead[]{}
\chead[]{}
\rhead[]{}
\renewcommand{\headrulewidth}{0pt}

\lfoot[]{}
\cfoot[]{}
\rfoot[]{}
\renewcommand{\footrulewidth}{0pt}

\fancypagestyle{plain}{
\fancyhead[L]{}
\fancyhead[C]{}
\fancyhead[R]{}
\fancyfoot[L]{}
\fancyfoot[C]{}
\fancyfoot[R]{}
\renewcommand{\headrulewidth}{0pt}
\renewcommand{\footrulewidth}{0pt}
}
\pagestyle{fancy}

\headheight 24.3pt

\newcommand{\grad}{$^{\circ}~$}
\newcommand{\tab}{\hspace{1cm}}

\begin{document}
\lhead[]{Universidad de La Frontera. INELE}
\chead[]{}
\rhead[]{\includegraphics[width=0.8cm]{./img/logo.png}}
\renewcommand{\headrulewidth}{0.5pt}

\lfoot[]{Protocolos de Comunicación}
\cfoot[]{}
\rfoot[]{\thepage}
\renewcommand{\footrulewidth}{0.5pt}

%--------------------------------------------------------------------------
\title{\includegraphics[width=4.7cm]{./img/logo.png} \\ \textbf{Informe Tarea N\grad 1}\\ %{\large Uso y configuración de una Raspberry}\\
{\small \textit{``Envío de mensajes"}}
}

\author{
\rule{10cm}{0.1mm} \\
Departamento de Ingeniería Eléctrica \\
\small Universidad de La Frontera \\
\rule{10cm}{0.1mm} 
}

\maketitle
\abstract{ Pequeño resumen del contenido del informe.}
\vfill
	\begin{flushright}
	\textbf{\uppercase\expandafter{Martín Canario}}\\
	\textbf{\uppercase\expandafter{Joaquín Argel}}\\
	\end{flushright}
\vskip 0.1in
	\begin{flushleft}
	\textbf{Profesor:} José Sánchez \\
	\end{flushleft}
\lstset{language=C, breaklines=true, basicstyle=\scriptsize}
\newpage\normalsize


\section{Comunicación.}
%Ejemplo
Cuando se quiere establecer una conexión entre 2 terminales, surgen diversos desafios a los cuales enfrentarse, uno de los principales es la desincronización entre ambas terminales, pues si simplemente iniciamos ambas terminales, puede darse el caso que coincidan en tiempos, pero conforme este pase, los errores de transmisión serán cada vez más notorios, es por ello que es necesario aplicar un plan para reducir el porcentaje de error en la comunicación.
\subsection{Comunicación Asincrónica.}
IBM{\cite{IBM}} define a la transmisión asíncrona como 'el proceso en el que los datos transmitidos se codifican con bits de inicio y de detención, que especifican el principio y el final de cada carácter.'

Nuestro código ocupa la transmisión asi

\section{Diseño del Protocolo de Comunicación.}
\subsection{CMD}
Nuestro CMD cuenta con 7 bits ya que en un inicio al contar los comandos que deberíamos usar, dimos que necesitábamos 6. Entonces pusimos 7 "por si acaso".
\subsection{Longitud}
Nuestro LNG cuenta con 6 bits debido a que considerábamos que necesitábamos una data medianamente grande para evitar problemas.
\subsection{Data}
Nuestra DATA es de 63 bits, esta se obtuvo calculando la sexta potencia de 2, tomando en cuenta el tamaño de LNG, esto nos da 64 bits y al restarle 1 bit tenemos nuestro tamaño de DATA.
\subsection{FCS}
FCS viene de las siglas Frame Check Sequence, en nuestro caso, FCS cuenta con 10 bits, ya que calculamos la cantidad máxima de bits activos, siendo en este caso 517, luego con logaritmos pudimos calcular que con \(2^{10}\) obtendríamos 1024, cantidad necesaria para el FCS, pues con \(2^{9}\) solo llegaba a 512.


\section{Manual de usuario.}

%Ejemplo
El siguiente es un código en C...
\begin{lstlisting}[frame=single]
#include <stdio.h>

int main() {
	printf("Hola mundo!!");
	return 0;
}
\end{lstlisting}

\section{Manual de Usuario.}
Como se aprecia en la figura~\ref{fig:logo}...

\begin{figure}[H]
\centering
\includegraphics[width=8cm]{./img/logo.png}
\caption{Título de la figura.}
\label{fig:logo}
\end{figure}

% Bibliografía.
%-----------------------------------------------------------------
\begin{thebibliography}{99}

\bibitem{Cd94} Autor, \emph{Título}, Revista/Editor, (año)
\bibitem{Wwp} Raspberry Pi Pinout, \emph{https://es.pinout.xyz/pinout/wiringpi}, (Marzo, 2018).
\bibitem{IBM} IBM, \emph{https://www.ibm.com/docs/es/aix/7.3?topic=synchronization-asynchronous-transmission},(Marzo,2023).

\end{thebibliography}

\end{document}
